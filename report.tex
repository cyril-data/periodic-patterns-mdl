\documentclass{report}
\usepackage{graphicx}
\usepackage{booktabs}
\usepackage{mathrsfs}
\usepackage{amsmath,amssymb} %,amsthm}
\usepackage{xifthen}% provides \isempty test
\usepackage{placeins}
\usepackage[margin=1.in,asymmetric]{geometry}

\newcommand{\dstname}[1]{\texttt{#1}}
\newcommand{\dstTZapX}{\dstname{3zap}}
\newcommand{\dstBugzX}{\dstname{bugzilla}}
\newcommand{\dstTZap}[1]{\dstname{3zap-{#1}}}
\newcommand{\dstBugz}[1]{\dstname{bugzilla-{#1}}}
\newcommand{\dstSacha}{\dstname{sacha}}
\newcommand{\dstSachaAR}[1]{\dstname{sacha-{#1}}}
\newcommand{\dstSachaG}[1]{\dstname{sacha-{\iAbs}-G{#1}}}
\newcommand{\dstSamba}{\dstname{samba}}
\newcommand{\dstUbi}{\dstname{ubiqLog}}
\newcommand{\dstUbiAR}[1]{\dstname{ubiqLog-{#1}}}
\newcommand{\dstUbiSAbs}[1]{\dstname{#1}}
\newcommand{\dstUbiSRel}[1]{\dstname{#1}}
\newcommand{\iRel}{rel}
\newcommand{\iAbs}{abs}

\DeclareMathOperator{\omed}{med}
\newcommand{\abs}[1]{\left\lvert#1\right\rvert}

\newcommand{\optspan}{\Delta}
\newcommand{\ABC}{\Omega}
\newcommand{\len}[1]{\abs{#1}}
\newcommand{\tspan}[1]{\optspan(#1)}
\newcommand{\seq}[1][]{%
   {\ifthenelse{\isempty{#1}}%
     {\ensuremath{S}}
     %{\ensuremath{S^{(#1)}}}
     {\ensuremath{S_{#1}}}
   }
}
\newcommand{\ccycle}{\ensuremath{\mathcal{C}}}
\newcommand{\cl}{\mathit{L}}
\newcommand{\prcCl}{\%\cl}

\newcommand{\collF}{\ccycle_{F}}
\newcommand{\collV}{\ccycle_{V}}
\newcommand{\collH}{\ccycle_{H}}
\newcommand{\collVH}{\ccycle_{V\!+H}}
\newcommand{\collS}{\ccycle_{S}}

\newcommand{\resSet}{\mathcal{R}}
\newcommand{\ratioClRC}[1]{\cl_{#1}\!:\!\resSet}
\newcommand{\ratioClR}{\cl\!:\!\resSet}
\newcommand{\clEmpty}{\cl(\emptyset, \seq)}
\newcommand{\clCC}{\cl(\ccycle, \seq)}

\newcommand{\nbTD}{m}
\newcommand{\nbV}{v}
\newcommand{\nbH}{h}
\newcommand{\nbS}{s}

\newcommand{\nbOTC}{c_{>3}}
\newcommand{\nbOmed}{c^{\text{M}}}
\newcommand{\nbOmax}{c^{+}}


\begin{document}
\input{./xps/table_datasets_agg}

\input{./xps/table_results_short}
\input{./xps/table_results_agg}

\begin{figure}[tbp]
\centering
\caption{Compression ratios for the sequences from the \dstUbiAR{\iAbs} dataset.}
\label{fig:xps-prcCl-ubi-abs}
\includegraphics[trim=20 0 40 0,clip,width=.49\textwidth]{./xps/fig_prcCL_UbiqLogISEAbs2-2}
\includegraphics[trim=20 0 40 0,clip,width=.49\textwidth]{./xps/fig_prcCL_UbiqLogISEAbs1-2}
\end{figure}



\begin{figure}[tbp]
  \centering
  \begin{tabular}{@{}cc@{}}
    \includegraphics[trim=5 0 20 0,height=3.5cm,clip]{./xps/fig_times(h)_all} &
    \includegraphics[trim=5 0 20 0,height=3.5cm,clip]{./xps/fig_times(m)_all} \\
  \end{tabular}
\caption{Running times for mining the different sequences (in hours, left) and zooming in on shorter sequences (in minutes, right).}
\label{fig:xps-times}
\end{figure}


\FloatBarrier
\begin{figure}[tbp] \centering
\caption{Compression ratios for planted and extracted pattern collections ($\prcCl_H$ and $\prcCl_F$, respectively) on synthetic sequences perturbed only by shift noise.}
\label{fig:synthe_1_cr}
\includegraphics[trim=0 0 0 0,clip,width=.8\textwidth]{./xps/synthe_S_scatter}
\end{figure}

\begin{figure}[tbp] \centering
\caption{Compression ratios for planted and extracted pattern collections ($\prcCl_H$ and $\prcCl_F$, respectively) on synthetic sequences perturbed by additive noise $(a, 0.1)$.}
\label{fig:synthe_2_cr}
\includegraphics[trim=0 0 0 0,clip,width=.8\textwidth]{./xps/synthe_V_scatter}
\end{figure}

\begin{figure}[tbp] \centering
\caption{Compression ratios for planted and extracted pattern collections ($\prcCl_H$ and $\prcCl_F$, respectively) on synthetic sequences perturbed by additive noise $(a, 0.5)$.}
\label{fig:synthe_3_cr}
\includegraphics[trim=0 0 0 0,clip,width=.8\textwidth]{./xps/synthe_W_scatter}
\end{figure}

\begin{figure}[tbp] \centering
\caption{Compression ratios for planted and extracted pattern collections ($\prcCl_H$ and $\prcCl_F$, respectively) on synthetic sequences containing interleaving.}
\label{fig:synthe_4_cr}
\includegraphics[trim=0 0 0 0,clip,width=.8\textwidth]{./xps/synthe_U_scatter}
\end{figure}

%%%%%%%%%%%%%%%%%%%%%%%%%%%%%%%%%%%%%%%%%
\begin{figure}[tbp] \centering
\caption{Differences in compression ratios for planted and extracted pattern collections ($\prcCl_H$ and $\prcCl_F$, respectively) on synthetic sequences perturbed only by shift noise.}
\label{fig:synthe_1}
\includegraphics[trim=50 20 100 20,clip,width=.8\textwidth]{./xps/synthe_S_box}
\end{figure}

\begin{figure}[tbp] \centering
\caption{Differences in compression ratios for planted and extracted pattern collections ($\prcCl_H$ and $\prcCl_F$, respectively) on synthetic sequences perturbed by additive noise $(a, 0.1)$.}
\label{fig:synthe_2}
\includegraphics[trim=50 20 100 20,clip,width=.8\textwidth]{./xps/synthe_V_box}
\end{figure}

\begin{figure}[tbp] \centering
\caption{Differences in compression ratios for planted and extracted pattern collections ($\prcCl_H$ and $\prcCl_F$, respectively) on synthetic sequences perturbed by additive noise $(a, 0.5)$.}
\label{fig:synthe_3}
\includegraphics[trim=50 20 100 20,clip,width=.8\textwidth]{./xps/synthe_W_box}
\end{figure}

\begin{figure}[tbp] \centering
\caption{Differences in compression ratios for planted and extracted pattern collections ($\prcCl_H$ and $\prcCl_F$, respectively) on synthetic sequences containing interleaving.}
\label{fig:synthe_4}
\includegraphics[trim=50 20 100 20,clip,width=.8\textwidth]{./xps/synthe_U_box}
\end{figure}

%%%%%%%%%%%%%%%%%%%%%%%%%%%%%%%%%%%%%%%%%
\begin{figure}[tbp] \centering
\caption{Compression ratios for planted and extracted pattern collections ($\prcCl_H$ and $\prcCl_F$, respectively) on synthetic sequences with multiple planted patterns.}
\label{fig:synthe_comb_cr}
\includegraphics[trim=0 0 0 0,clip,width=.8\textwidth]{./xps/synthe_comb_scatter}
\end{figure}

\begin{figure}[tbp] \centering
\caption{Differences in compression ratios for planted and extracted pattern collections ($\prcCl_H$ and $\prcCl_F$, respectively) on synthetic sequences with multiple planted patterns.}
\label{fig:synthe_comb}
\includegraphics[trim=0 0 80 20,clip,width=.8\textwidth]{./xps/synthe_comb_box}
\end{figure}

\FloatBarrier

\input{./xps/table_datasets_stats}

\input{./xps/table_results_all_long}

\FloatBarrier

\begin{figure}
\centering
\caption{Compression ratios for \dstTZapX{}, \dstBugzX{} and \dstSamba{} sequences.}
\label{fig:xps-prcCl-other}
\includegraphics[trim=0 0 40 30,clip,width=.7\textwidth]{./xps/fig_prcCL_Other}
\end{figure}

\begin{figure}
\centering
\caption{Compression ratios for \dstSachaAR{\iAbs} sequences with various time granularities.}
\label{fig:xps-prcCl-sacha}
\includegraphics[trim=20 0 40 0,clip,width=.8\textwidth]{./xps/fig_prcCL_Sacha}
\end{figure}

\begin{figure}
\centering
\caption{Compression ratios for the sequences from the \dstUbiAR{\iAbs} dataset.}
\label{fig:xps-prcCl-ubi-abs}
\includegraphics[trim=30 0 40 30,clip,width=.8\textwidth]{./xps/fig_prcCL_UbiqLogISEAbs}
\end{figure}

\begin{figure}
\centering
\caption{Compression ratios for the sequences from the \dstUbiAR{\iRel} dataset.}
\label{fig:xps-prcCl-ubi-rel}
\includegraphics[trim=30 0 40 30,clip,width=.8\textwidth]{./xps/fig_prcCL_UbiqLogISRel}
\end{figure}

\FloatBarrier

\begin{figure}[tbp]
  \centering
  \begin{tabular}{@{}ccc@{}}
    \includegraphics[trim=5 0 20 0,width=.32\textwidth,clip]{./xps/fig_times(h)_details} &
    \includegraphics[trim=5 0 20 0,width=.32\textwidth,clip]{./xps/fig_times(m)_details} &
    \includegraphics[trim=5 0 20 0,width=.32\textwidth,clip]{./xps/fig_times(s)_details} \\
    \multicolumn{3}{c}{\hfill \includegraphics[trim=100 133 10 222,width=.45\textwidth,clip]{./xps/fig_times(l)_details} \hfill
    \includegraphics[trim=100 45 10 310,width=.45\textwidth,clip]{./xps/fig_times(l)_details} \hfill} \\
  \end{tabular}
\caption{Running times for sequences from the different datasets, in hours (left) and zoomed-in in minutes (center) and seconds (right).}
\label{fig:xps-times-details}
\end{figure}

\end{document}
%%% Local Variables:
%%% mode: latex
%%% TeX-master: t
%%% End:
